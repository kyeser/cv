\ifcase\pdfoutput
\documentclass[dvipdfmx,a4paper]{article}
\else
\documentclass[a4paper,10pt,draft]{article}
\fi
\usepackage{geometry}
\usepackage[cjk,hangul,usedotemph]{kotex}
\usepackage{xcolor,hologo}
%\usepackage[pdfencoding=auto,
%  pdftitle={이력서},
%  pdfauthor={계희승},
%  pdfkeywords={CJK, LaTeX, Korean, ko.TeX}
%  ]{hyperref}
\usepackage{hanging}
\usepackage{textcomp}
\usepackage{amsmath,amssymb,latexsym}
\def\cs#1{\texttt{\color{teal}\char92 \chardef\{=123 \chardef\}=125 #1}}
\def\koTeX{\textsf{k}\kern-.1em\textit{o}.\kern-.1667em\TeX}
\def\cjkko{\mbox{CJK-\textsf{k}\kern-.1em\textit{o}}}
\usepackage{fancyhdr}
\setlength{\headheight}{15.2pt}
\pagestyle{fancy}
\linespread{1.3}

\title{이력서}
\author{계희승}
\date{\today}

\begin{document}
%\maketitle
  \renewcommand{\headrulewidth}{0pt}
  \fancyhf{}
  \fancyhead[LE,LO]{\small 이력서(계희승)}
  \fancyfoot[RE,RO]{\small \thepage}
  
  \hspace*{-0.8cm}
  \begin{tabular}{p{7.6cm} r}
    {\large 계희승(桂熙承, Hee Seng Kye)} & 한양대학교 음악연구소\\
    undoingmusicology.com & 04763 서울특별시 성동구 왕십리로 222\\
    & Tel. 02 2220 1512 / Fax. 02 2281 1512\\
    & hskye@hanyang.ac.kr / mrc.hanyang.ac.kr
  \end{tabular}
  
  \vspace{15mm}
  
  \section*{\normalsize 현재}
  
  \hspace*{-0.25cm}
  \begin{tabular}{p{4.0cm} l}
    2019년 9월 $\sim$ 현재 & 한양대학교 음악연구소 연구원\\
    & 참여과제: `소리와 청취의 정치학'(인문사회연구소지원사업)
  \end{tabular}
  
  \vspace*{2.5mm}
  
  \section*{\normalsize 학력}
  
  \hspace*{-0.25cm}
  \begin{tabular}{p{4.0cm} p{10.0cm}}
    2010년 9월 $\sim$ 2015년 11월 & 홍콩대학교 대학원 철학박사(음악학)\\
    & The University of Hong Kong (Ph.D. in Musicology)\\
    & 지도교수: 김 연(Youn Kim)\\
    & 학위논문: The Third Voice: Anima as Drama in Mozart’s Operas\\
    & 심사위원: Stephen Matthews, chairman; Su Yin Mak, external;\\
    & \hspace*{14mm} Giorgio Biancorosso \& Daniel Chua, internal\\[2mm]
    
    2008년 3월 $\sim$ 2010년 3월 & 한양대학교 일반대학원 음악학과 박사과정(서양음악이론) 수료\\
    & 지도교수: 박재성\\[2mm]
    
	2003년 8월 $\sim$ 2005년 6월 & 뉴욕시립대학교 퀸스컬리지 예술석사(작곡)\\
    & Queens College, CUNY (M.A. in Composition)\\
	& 지도교수: 제프 니콜스(Jeff Nichols)\\[2mm]

    1999년 9월 $\sim$ 2003년 5월 & 줄리아드 학교 음악학사(작곡)\\
    & The Juilliard School (B.M. in Composition)\\
    & 지도교수: 밀튼 배빗(Milton Babbitt)
  \end{tabular}
  
  \vspace{5mm}
  
  \section*{\normalsize 연구\textperiodcentered 강의 분야}
  
  \noindent \hspace{2mm} \textbullet \hspace{2mm} 음악 형식과 분석
  
  \noindent \hspace{2mm} \textbullet \hspace{2mm} 영화\textperiodcentered 게임 음악의 현상학
  
  \noindent \hspace{2mm} \textbullet \hspace{2mm} 소리와 청취의 정치학
  
  \noindent \hspace{2mm} \textbullet \hspace{2mm} 음악과 질병, 장애의 문화사
  
  \section*{\normalsize 연구경력}
  
  \hspace*{-0.25cm}
  \begin{tabular}{p{4.0cm} l}
    2015년 9월 $\sim$ 2018년 8월 & 한양대학교 음악연구소 전임연구원\\
    & 참여과제: ‘사이’ 공간의 소리환경 연구(중점연구지원사업)\\
    2010년 5월 $\sim$ 2013년 4월 & 서울대학교 음악대학\textperiodcentered 서울대학교병원 MEG 센터 공동연구\\
    & 참여과제: 음악이론의 신경과학적 기반 연구(연구보조원)
  \end{tabular}
  
  \vspace{5mm}
  
  \section*{\normalsize 출판목록}

  \subsection*{\small 논문}
  \hspace*{-0.25cm}
  \begin{tabular}{p{3.0cm} p{11cm}}
    2018년 & “Excavating the History of Electroacoustic Music in Korea,
    1966–2016.” \textit{Contemporary Music Review} 37 (1–2): 174–87. (공저자: 임종우)\\[2mm]
    
    2017년 & “벤자민 브리튼과 나사의 회전 소리, 혹은 ‘해석’의 전회.” 『음악이론포럼』 제24--2호, pp. 71$\sim$94.\\[2mm]
    
    2017년 & “바실리오와 피가로는 무엇을 들었는가? 오페라, 혹은 엿듣기의 예술.” 『음악논단』 제38집, pp. 1$\sim$33.\\[2mm]
    
    2017년 & “비텔리아의 양가감정, 마르케티의 주이상스.” 『음악이론연구』 제28집, pp. 44$\sim$67.\\[2mm]
    
    2017년 & “세스토의 아니마, 비텔리아의 아니무스: 모차르트 론도 아리아에서 듣는 내면의 소리.” 『서양음악학』 제20--1호, pp. 71$\sim$98.\\[2mm]
    
    2015년 & “피가로는 정말 수잔나에게 화가 난 것일까? 음악분석으로 되읽는 오페라.” 『음악이론연구』 제25집, pp. 8$\sim$34.
  \end{tabular}
  
  \subsection*{\small 장章}
  \hspace*{-0.25cm}
  \begin{tabular}{p{3.0cm} p{11.0cm}}
    2020년 & “Excavating the History of Electroacoustic Music in Korea,
    1966–2016.” In \textit{Electroacoustic Music in East Asia}, edited by Marc Battier and Kenneth Fields, 172--85. New York: Routledge, 2020. (공저자: 임종우; 2018년 논문의 리프린트)\\[2mm]
    
    2013년 & “화성진행의 통사론, 기대감과 실현: 신경과학적 연구의 가능성을 찾아.” 『음악의 지각과 인지 II』. 이석원 책임편집, pp. 225$\sim$89. 서울: 음악세계. (공저자: 이석원, 서준교, 박정미, 김찬희, 배다혜)
  \end{tabular}
  
  \subsection*{\small 번역}
  \hspace*{-0.25cm}
  \begin{tabular}{p{3.0cm} p{11.0cm}}
    2013년 & “음악\textperiodcentered 동작\textperiodcentered 마림바: 청중에게 음악적 표현을 전달하는 데 있어 동작과 몸짓이 미치는 영향.” 『음악의 지각과 인지 II』. 이석원 책임편집, pp. 307$\sim$24. 서울: 음악세계. 원문: “Music, Movement and Marimba: An Investigation of the Role of Movement and Gesture in Communicating Musical Expression to an Audience” by Mary Broughton and Catherine Stevens, \textit{Psychology of Music} 37, no. 2 (2009): 137–53.
  \end{tabular}
  
  \vspace{5mm}
  
  \section*{\normalsize 강의경력}
  
  \hspace*{-0.25cm}
  \begin{tabular}{p{3.0cm} p{11.0cm}}
    2021년 $\sim$ 현재 & 국민대학교 예술대학 음악학부 및 일반대학원 음악학과\\
    & 음악분석 1 \& 2(피아노), 피아노 세미나 III, 피아노작품분석기법,\\
    & 현대음악작곡가집중연구\\[1mm]
    
    2019년 $\sim$ 현재 & 서울대학교 음악대학\\
    & 음악분석 2, 3, 4(피아노), 음악분석 2, 3, 4(현악), 이론 1 \& 2(국악)\\[1mm]
    
    2015년 $\sim$ 현재 & 한양대학교 음악대학 및 일반대학원\\
    & 4차 산업혁명과 음악, 관현악문헌 1, 음악 형식과 분석 1 \& 2, 음악과 문화,\\
    & 음악사원전읽기, 음악학세미나, 전공실기\\[1mm]
    
    2016년 $\sim$ 2020년 & 성신여자대학교 음악대학 및 대학원\\
    & 음악문화연구, 음악현장연구, 전공이론\\[1mm]
    
    2018년 $\sim$ 2019년 & 추계예술대학교 음악대학\\
    & 서양음악사 1 \& 2(성악과, 작곡과), 음악사 1 \& 2(관현악과)\\[1mm]
    & 추계예술대학교 문화예술경영대학원\\
    & 음악연구\\[1mm]
    
    2018년 $\sim$ 2019년 & 동덕여자대학교 일반대학원\\
    & 음악과 문화, 음악사세미나, 작곡가연구, 전공연구\\[1mm]
    
    2016년 & 연세대학교 음악대학원\\
    & 고전파 음악 Music of Classical Period(영어강의)\\
    & 피아노 문헌 Seminar in Piano Literature I(영어강의)\\[1mm]
    
    2015년 & 경북대학교 예술대학, 음악이론세미나\\[1mm]
    & 홍콩대학교 인문대 음악과, Topics in Western Music History III(영어강의)\\[1mm]
    
    2010년 & 한세대학교 음악학부, 음악분석, 화성학\textperiodcentered 대위법 1\\[1mm]
    
    & 한양대학교 음악대학 작곡과, 음악이론 1 \& 5\\[1mm]
    
    2009년 & 한국예술종합학교 음악원, 집합이론 특강\\
  \end{tabular}
  
  \section*{\normalsize 학술대회 발표}
  
  \noindent 2022년 3월 24$\sim$27일 하와이 호놀룰루\\
  AAS 2022 Annual Conference\\
  발표논문 “The \textit{Waesaek} Controversies, or the Sound of Modernity in Park Chan-wook’s \textit{The Handmaiden},” Panel session, “Tempting Tunes: The Soundscape of South Korean Popular Culture”\\
  
  \noindent 2021년 11월 20일 화상회의 플랫폼 Zoom\\
  한국서양음악학회 제89차 학술포럼\\
  라운드테이블 ``한국의 (서양)음악학, 어디로 갈 것인가''\\
  
  \noindent 2021년 10월 28$\sim$31일 프랑스 라로셸(온라인)\\
  30th AKSE Conference\\
  발표논문 “Hymn, Lullaby, and ‘Happy Birthday’: Park Chan-wook’s \textit{Lady Vengeance} as a Musical Commentary on Protestant Christianity in Korea,” Panel session, “The Current Soundscape of Korea: \textit{Trot}, K-Pop, Film Music, and Art Music”\\
  
  \noindent 2021년 6월 5일 화상회의 플랫폼 Zoom\\
  한양대학교 음악연구소 인문사회연구소지원사업 제2회 학술대회 『소리와 청취의 정치학 II』\\
  발표논문 “누구의 꿈인가? 넷플릭스 드라마 《브리저튼》과 ‘클래식 커버’ 음악의 젠더화된 소리들”\\
  
  \noindent 2020년 11월 19$\sim$21일 화백컨벤션센터(HICO), 경상북도 경주시\\
  제6회 세계인문학포럼(The 6th World Humanities Forum)\\
  발표논문 “Meet the ‘Musical’ Hulk: Tracing New American Masculinity in \textit{Avengers: Age of Ultron},” Panel session, “Classes of Sound, or on the Politics of Sonic Identity”\\
  
  \noindent 2020년 10월 17일 화상회의 플랫폼 Zoom\\
  한양대학교 음악연구소 인문사회연구소지원사업 제1회 학술대회 『소리와 청취의 정치학 I』
\\
  발표논문 “마지막 오르페오의 노래: 코지마 히데오의 《데스 스트랜딩》과 듣기의 쓸모”\\
  
  \noindent 2019년 10월 18$\sim$20일 중국 쑤저우대학교(苏州大学)\\    
  The 5th Biennial Conference of the East Asian Regional Association of IMS\\
  발표논문 “Rehabilitating Gemma, or Hearing the Voices of an Empty Womb in Donizetti’s \textit{Gemma di Vergy}”\\
  
  \noindent 2019년 8월 24일 한국방송통신대학교\\
  한국프랑스고전문학회 제77차 연구발표회\\
  발표논문 “음악적 피가로의 탄생: 오페라, 혹은 반복의 예술”\\
  
  \noindent 2018년 5월 24$\sim$27일 뉴욕대학교(New York University)\\
  Music \& The Moving Image XIII\\
  발표논문 “The Fellowship of the Sound(tracks): How the Regalia Became a Place to Unwind in \textit{Final Fantasy XV}”\\
  
  \noindent 2018년 3월 30$\sim$31일 한양대학교\\
  Rethinking Sound 2018\\
  발표논문 “Why was Bruce Listening to ‘Casta diva’? Soundtrack as a Sonic/Sonified Conscience in \textit{Avengers: Age of Ultron}”\\
  
  \noindent 2017년 6월 10일 한양대학교\\
  한양대학교 음악연구소 대학중점연구지원사업 제3회 학술대회 『다시 듣다』\\
  발표논문 “그래 봤자 음악, 그래도 음악: 게임 음악(학)이 들려주는 소리들”\\
  
  \noindent 2017년 3월 19$\sim$23일 동경예술대학(Tokyo University of the Arts)\\
  20th Quinquennial Congress of the International Musicological Society\\
  발표논문 “Soundscape of the Future in Sci-fi Film: The ‘Aural’ Gaze and the Dissolution of Subjectivity”\\
  
  \noindent 2016년 11월 19일 성신여자대학교\\
  한국서양음악학회 제80차 학술포럼\\
  발표논문 “‘언제 다시 우리 셋이 만날까?’ 밀튼을 기억하며”\\
  
  \noindent 2016년 7월 24$\sim$29일 서울대학교\\
  제20차 세계미학자대회(20th International Congress of Aesthetics)\\
  라운드테이블 “Politics of Noise: The Soundscape of 2016 General Election Campaign in Korea”\\
  
  \noindent 2016년 6월 1$\sim$4일 오루후스대학교(Aarhus University, Denmark)\\
  Sound Art Matters, International Conference\\
  발표논문 “(Re)sounding the Virtual: Hearing the Voice of Hatsune Miku”\\
  
  \noindent 2015년 11월 21일 성신여자대학교\\
  한국서양음악학회 제78차 학술포럼\\
  발표논문 “세스토의 아니마, 비텔리아의 아니무스: 모차르트 론도 아리아에서 듣는 내면의 소리”\\
  
%  \noindent 2015년 8월 27일 연세대학교 음악대학\\
%  2015년 한국서양음악이론학회(KSMT) 학술포럼\\
%  발표주제 “보마스셰의 피가로, 다 폰테의 피가로, 모차르트의 피가로: 음악분석으로 되읽는 모차르트 오페라”\\
  
  \noindent 2013년 10월 18$\sim$20일 대만국립대학교\\
  The 2nd Biennial Conference of the East Asian Regional Association of IMS\\
  발표주제 “Neural Responses to Tonality and Atonality: How Our Brains Perceive the Difference”\\
  
%  \noindent 2013년 4월 12$\sim$13일 뉴욕시립대학교\\
%  2013 Graduate Center GSIM (Graduate Students in Music) Conference\\
%  발표논문 “Mozart the Philosopher: Aside, Silence, and Time in \textit{La clemenza di Tito}”\\
  
  \noindent 2012년 3월 9$\sim$11일 홍콩대학교\\
  Music and the Body 2012\\
  공동발표 “Harmonic Meaning as Musical Semantics (and the Role of Right Hemisphere): Evidence from Magnetoencephalography”
  
%  \noindent 2011년 9월 16$\sim$18일 서울대학교 음악대학\\
%  Inaugural Conference of the East Asian Regional Association of IMS\\
%  발표논문 “Susanna’s Choices: Anima as Drama in Mozart’s \textit{Le nozze di Figaro}”
  
%  \noindent 2009년 6월 24일 서울대학교 음악대학\\
%  서울대학교 음악대학 서양음악연구소 학술포럼\\
%  발표주제 “A ‘Tutorial’ on K-nets under Multiplicative Operations”\\
  
%  \noindent 2009년 6월 13일 서울대학교 음악대학 시청각실\\
%  한국음악지각인지학회 제43차 학술포럼\\
%  공동발표 “화성진행의 통사론, 기대감과 실현: 신경과학적 연구의 가능성을 찾아”\\
  
%  \noindent 2008년 9월 20일 서울대학교 음악대학 시청각실\\
%  한국음악지각인지학회 10주년 기념 학술대회\\
%  발표논문 “Klumpenhouwer Networks in T/M Group: An Application of Multiplicative Operations to Recursive System”
  
  \vspace{2.5mm}
  
  \section*{\normalsize 대표작품}
  
  \hspace*{-0.25cm}
  \begin{tabular}{p{1.5cm} p{12.5cm}}
    2003년 & \textit{Handel’s Invective} for guitar and computer-generated sound (Soichi Muraji, guitar; Tokyo Cultural Center, Japan, 2004)\\[0.5mm]
    
    2003년 & \textit{Cantus Subtilior: 370 cantus in vocis duodecim} for four
    players\\[0.5mm]
    
    2003년 & \textit{A Measure of Hans Weisse} for twelve violins\\[0.5mm]
    
    2002년 & \textit{The Fugue of Art} for violin and orchestra quartet\\[0.5mm]
    
    2002년 & \textit{Corelli’s Protest} for solo piano\\[0.5mm]
    
    2002년 & \textit{A Transfigured Offering} for two chamber orchestras\\[0.5mm]
    
    2000년 & String Quartet No. 1\\[0.5mm]
    
    1999년 & \textit{In Memoriam Isang Yun} for orchestra (Jeffrey Milarsky, conductor; Juilliard Symphony, Alice Tully Hall, Lincoln Center, New York, United States, 2000)
  \end{tabular}
  
  \vspace{2.5mm}
  
  \section*{\normalsize 기타}
  
  \hspace{2mm} \textbullet \hspace{2mm} International Musicological Society East Asia Regional Association (IMSEA) 2022 조직위원
  
  \noindent \hspace{2mm} \textbullet \hspace{2mm} IMSEA Virtual Conference 2021 프로그램 구성위원
  
  \noindent \hspace{2mm} \textbullet \hspace{2mm} American Musicological Society 정회원
  
  \noindent \hspace{2mm} \textbullet \hspace{2mm} Society for Music Theory 정회원
  
  \noindent \hspace{2mm} \textbullet \hspace{2mm} 한국서양음악학회 이사(학술위원)
  
  \noindent \hspace{2mm} \textbullet \hspace{2mm} 한국서양음악이론학회 이사
  
  \noindent \hspace{2mm} \textbullet \hspace{2mm} KBS 클래식FM 〈KBS 음악실〉 ‘계희승의 음악 허물기’ 출연(매주 월요일)

  \noindent \hspace{2mm} \textbullet \hspace{2mm} Python, R, HTML5/CSS 구사 가능

  \vspace{2.5mm}
  
  \section*{\normalsize 병역사항(병장만기전역)}
  
  \hspace*{-0.25cm}
  \begin{tabular}{p{5.5cm} l}
    2005년 8월 22일 $\sim$ 2007년 8월 21일 & 육군교육사령부(군사특기: 어학--번역)
  \end{tabular}
\end{document}